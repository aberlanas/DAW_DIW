% Created 2019-11-25 lun 19:33
\documentclass[11pt]{article}
\usepackage[utf8]{inputenc}
\usepackage[T1]{fontenc}
\usepackage{fixltx2e}
\usepackage{graphicx}
\usepackage{longtable}
\usepackage{float}
\usepackage{wrapfig}
\usepackage{rotating}
\usepackage[normalem]{ulem}
\usepackage{amsmath}
\usepackage{textcomp}
\usepackage{marvosym}
\usepackage{wasysym}
\usepackage{amssymb}
\usepackage{hyperref}
\tolerance=1000
\usepackage[newfloat]{minted}
\hypersetup{colorlinks=true,linkcolor=blue}
\author{Angel Berlanas Vicente}
\date{\today}
\title{Instalación de Wordpress}
\hypersetup{
  pdfkeywords={},
  pdfsubject={},
  pdfcreator={Emacs 25.2.2 (Org mode 8.2.10)}}
\begin{document}

\maketitle
\tableofcontents


\section{Antes de la instalación}
\label{sec-1}

\href{https://wordpress.org/support/article/before-you-install/}{Requisitos}

\subsection{PHP 7.3 y Apache}
\label{sec-1-1}

\href{https://r00t4bl3.com/post/how-to-install-php-7-3-on-ubuntu-18-04-bionic-beaver}{Instalacion
via PPA}

\begin{minted}[]{bash}
sudo add-apt-repository ppa:ondrej/php
sudo add-apt-repository ppa:ondrej/apache2
sudo apt update
sudo apt install php7.3
sudo apt install apache2
\end{minted}


\subsection{Configurando Apache2 y php7.3}
\label{sec-1-2}

Comprobamos qué módulos están habilitados en nuestro Apache, ejecutando \verb~a2query -m~, que nos devuelve una lista de los módulos, así como \emph{quién} los ha habilitado.

\begin{minted}[]{bash}
sudo a2dismod php7.2
sudo a2enmod php7.3
sudo systemctl restart apache2
\end{minted}

\subsection{Instalación de MySQL o MariaDB}
\label{sec-1-3}

\begin{figure}[htb]
\centering
\includegraphics[width=.9\linewidth]{./imgs/mariadb-vs-mysql.png}
\caption{MariaDB o MySQL}
\end{figure}

Si elegimos MariaDB, debemos tener en cuenta que se trata de un reemplazo
\emph{binario} de todos los comandos de \emph{mysql}.

\begin{minted}[]{bash}
apt install mariadb-server-10.1
\end{minted}

Si da problemas con una instalación previa de \verb~mysql~, seguir los pasos que
aparecen aquí : \href{https://mariadb.com/kb/en/library/the-community-mariadb-troubles-only-running-after-reboot-times-out-when-try/}{MariaDB}.

\subsection{Creación de la BD y el Usuario}
\label{sec-1-4}

\begin{minted}[]{sql}
create database wordpress;
grant all privileges on wordpress.* to "wpadmin"@"localhost" identified by "wp4dm1n";
flush privileges;
\end{minted}


\section{Instalación de Wordpress}
\label{sec-2}

\begin{enumerate}
\item Descargar el fichero zip desde la pagina oficial \href{https://wordpress.org/download/}{Wordpress}
\item Copiarlo a \verb~/var/www/html/~ y descomprimirlo.
\item (Opcional) Mover el directorio a \verb~wordpress-VERSION~ y crear un enlace
simbólico.
\item Seguir el asistente con el navegador.
\end{enumerate}

Datos relevantes para el Wordpress del ejemplo:

\begin{center}
\begin{tabular}{ll}
Usuario & Password\\
cthulhu & 49R'lyeh128\\
\end{tabular}
\end{center}

\section{Desarrollo de un tema}
\label{sec-3}

Podemos encontrar algunos recursos interesantes en la
\href{https://developer.wordpress.org/themes/getting-started/setting-up-a-development-environment/}{Página Oficial} 
% Emacs 25.2.2 (Org mode 8.2.10)
\end{document}
