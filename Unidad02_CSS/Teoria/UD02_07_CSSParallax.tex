% Created 2019-11-11 lun 15:32
\documentclass[11pt]{article}
\usepackage[utf8]{inputenc}
\usepackage[T1]{fontenc}
\usepackage{fixltx2e}
\usepackage{graphicx}
\usepackage{longtable}
\usepackage{float}
\usepackage{wrapfig}
\usepackage{rotating}
\usepackage[normalem]{ulem}
\usepackage{amsmath}
\usepackage{textcomp}
\usepackage{marvosym}
\usepackage{wasysym}
\usepackage{amssymb}
\usepackage{hyperref}
\tolerance=1000
\usepackage[newfloat]{minted}
\usepackage{minted}
\usepackage{xcolor}
\usemintedstyle{monokai}    %% sets default for all source-code blocks
\definecolor{bg}{HTML}{555555}
\setminted{bgcolor=bg}
\hypersetup{colorlinks=true,linkcolor=black}
\author{Angel Berlanas}
\date{\today}
\title{UD02 - Paralaje en CSS}
\hypersetup{
  pdfkeywords={},
  pdfsubject={},
  pdfcreator={Emacs 25.2.2 (Org mode 8.2.10)}}
\begin{document}

\maketitle
\tableofcontents



\section{Efectos de Paralaje en CSS}
\label{sec-1}

\subsection{¿En qué consiste?}
\label{sec-1-1}

El efecto de Paralaje es la simulación en un entorno 2D de que estamos
observando a través de una ventana un mundo en 3 Dimensiones, donde dependiendo
de la lejania del objeto respecto al observador, este se desplaza a diferente
velocidad. 

Es el efecto de mirar a través de la ventana de un tren, los objetos que se
encuentran próximos a nosotros parecen desplazarse a una velocidad mayor,
mientras que los más lejanos apenas se mueven.

\subsection{Maneras de implementarlo}
\label{sec-1-2}

Se tratar de algo bastante complejo respecto a lo visto anteriormente en
clase, si buscamos en \emph{Internet} \verb~efecto de paralaje JS~, veremos multitud de
librerias ya realizadas que con mayor o menor éxito y posibilidad de
configuración se adaptan \emph{a nuestras necesidades}.

Con el objetivo de plantear las \textbf{bases} de como funciona este tipo de efecto,
vamos a decantarnos por una implementación del efecto en CSS Puro (\emph{Pure CSS}),
que nos permitirá no \emph{distraernos} con el código en \verb~JS~. Sin duda a medida que
el desarrollador en ciernes (el alumn@), vaya desarrollando sus habilidades,
buscará soluciones más complejas y vistosas que le permitirán un mayor
despliegue de su talento. Por ahora\ldots{}CSS es nuestro amigo.

\newpage
\subsection{HTML}
\label{sec-1-3}

\begin{minted}[]{html}
<body>
	<div class="navbar"><span>Paralaje en CSS</span></div>  
	<div class="parallax-wrapper">
	    <div class="content">
		<p>Lorem ipsum dolor sit amet,... 
		   ....
		   Donec in justo eu ligula semper.</p>
	    </div>
	</div>
	<div class="regular-wrapper">
	    <div class="content">
		<p>Lorem ipsum dolor sit amet,...
		   .... 
		   Donec in justo eu ligula semper.</p>
	    </div>
	</div>
</body>
\end{minted}

Comenzamos con poner el fondo al 100\%.

\begin{minted}[]{css}
.parallax-wrapper {
    width: 100vw;
    height:100vh;
    padding-top:20vh;
    box-sizing: border-box;
}
.regular-wrapper {
    width: 100vw;
    height:100vh;
    padding-top:20vh;
    background-image: url("/your-bkg.png");
}
.content {
    margin: 0 auto;
    padding: 50px;
    width: 50%;
    background: #aaa;
}
\end{minted}

Pasamos a ponerlo absoluto


\begin{minted}[]{css}
.parallax-wrapper::before {
    content:"";
    width: 100vw;
    height: 100vh;
    top:0;
    left:0;
    background-image: url("/ruta-al-background.png");
    position: absolute;
    z-index: -1;
}
\end{minted}


Vamos añadiendo a los contenidos el código \verb~CSS~:

\begin{minted}[]{css}
html {
    overflow: hidden;
}
body {
    height: 100vh;
    perspective: 1px;
    transform-style: preserve-3d;
    overflow-x:hidden;
    overflow-y:auto;
}
.parallax-wrapper {
    transform-style: preserve-3d;
}
\end{minted}


Transformamos y escalamos antes.

\begin{minted}[]{css}
.parallax-wrapper::before {
    transform:translateZ(-1px) scale(2);
}
\end{minted}

Antes de esto modificamos el efecto.

\begin{minted}[]{css}
.parallax-wrapper::before {
    z-index: -1;
}
.regular-wrapper {
    z-index: 2;
    position: relative;
}
\end{minted}
% Emacs 25.2.2 (Org mode 8.2.10)
\end{document}
