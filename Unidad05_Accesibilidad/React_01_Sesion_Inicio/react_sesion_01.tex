% Created 2020-01-29 mié 18:56
% Intended LaTeX compiler: pdflatex
\documentclass[11pt]{article}
\usepackage[utf8]{inputenc}
\usepackage[T1]{fontenc}
\usepackage{graphicx}
\usepackage{grffile}
\usepackage{longtable}
\usepackage{wrapfig}
\usepackage{rotating}
\usepackage[normalem]{ulem}
\usepackage{amsmath}
\usepackage{textcomp}
\usepackage{amssymb}
\usepackage{capt-of}
\usepackage{hyperref}
\usepackage[newfloat]{minted}
\hypersetup{colorlinks=true,linkcolor=black}
\author{Angel Berlanas}
\date{\today}
\title{UD05 - React Session - Vol 1}
\hypersetup{
 pdfauthor={Angel Berlanas},
 pdftitle={UD05 - React Session - Vol 1},
 pdfkeywords={},
 pdfsubject={},
 pdfcreator={Emacs 26.3 (Org mode 9.1.9)}, 
 pdflang={English}}
\begin{document}

\maketitle
\tableofcontents


\section{Disclaimer}
\label{sec:orgddb02e2}

Todos los apuntes que van a tener lugar en estas carpetas y sesiones 
han sido escritos con el objetivo de presentar a los alumnos de DAW una visión 
global y desde la humildad de aquel que se acerca a la montaña de conocimiento 
y con pocas herramientas comienza la escalada.

No todos los conceptos estan dominados, muchas de las tecnologías están fuera
de lo que llamaría \emph{mi zona de comfort}, pero sin duda vale la pena practicar
y prepararse todo esto\ldots{} 

Como dicen en algunas conferencias, vale la pena tener creatividad que
comodidad.

No espero más para los alumnos y los profesores que lleguen a leer esto, que
les ayude en la medida de lo posible.

\section{¿Qué es React?}
\label{sec:org47a3c40}

La definición que nos aporta DuckDuckGo es: 

\emph{React es una biblioteca Javascript de código abierto para crear}
\emph{interfaces de usuario con el objetivo de animar al desarrollo /
/de aplicaciones en una sola  página. Es mantenido por Facebook,}
\emph{Instagram y una comunidad de}
\emph{desarrolladores independientes y compañías.}

Que sea de código abierto y sea mantenida por \emph{Facebook} es algo bastante
curioso, pero la \emph{comunidad} es la \emph{comunidad}.

\begin{figure}[htbp]
\centering
\includegraphics[width=.9\linewidth]{./imgs/react-logo.png}
\caption{Logo de React}
\end{figure}

\section{Muchas maneras de acercarse a React}
\label{sec:orgc47f906}

Existen muchas, tal y como aparecen descritas en su página web:

\href{https://reactjs.org/}{Pagina Oficial de ReactJS}

De todas ellas, elegiremos que \emph{creo} más adecuada, que es la creación de una
página-app que será autocontenida.

\href{https://reactjs.org/docs/create-a-new-react-app.html}{Create a New React App}

Pasamos al turrón.

\section{Creando nuestra nueva App React}
\label{sec:org0fcb6d0}

\subsection{Motivos por los que esto mola}
\label{sec:org9c386a0}

Scaling to many files and components.
Using third-party libraries from npm.
Detecting common mistakes early.
Live-editing CSS and JS in development.
Optimizing the output for production.







\subsection{Los Toolchains esos grandes desconocidos}
\label{sec:org1375f3a}

\textbf{Un gestor de paquetes} : \texttt{npm} \^{}\_\^{}
\textbf{Un bundler}: \texttt{webpack} o \texttt{parcel}, nos permite programar de manera modular y
luego juntarlo todo como \$DEITY manda.
\textbf{Un compilador} : \texttt{babel} nos permite programar con JS Moderno y luego que
funcione en navegadores pleistozenicos.

\href{https://reactjs.org/docs/create-a-new-react-app.html\#creating-a-toolchain-from-scratch}{Creando un Toolchain desde la rascadura}


\subsection{Inicializando nuestra APP}
\label{sec:orgda8b651}

\begin{minted}[]{bash}
npx create-react-app my-app
cd my-app
npm start
\end{minted}

Una vez realizado esto, nos pone en marcha un node en nuestro
\emph{localhost:3000}

Y amablement nos indica que comenzemos a editarla.
\end{document}
